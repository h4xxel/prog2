%% En enkel mall för att skapa en labb-raport.
\documentclass[a4paper, 11pt]{article}
\usepackage[utf8]{inputenc} 
\usepackage[swedish]{babel}
\usepackage{verbatim}
\usepackage{listings}
\usepackage{color}

\definecolor{dkblue}{rgb}{0,0,0.5}
\definecolor{dkyellow}{rgb}{0.5,0.5,0}
\definecolor{gray}{rgb}{0.5,0.5,0.5}
\definecolor{dkgreen}{rgb}{0,0.5,0}

\lstset{ %
    backgroundcolor=\color{white},   % choose the background color; you must add \usepackage{color} or \usepackage{xcolor}
    basicstyle=\footnotesize,        % the size of the fonts that are used for the code
    breakatwhitespace=false,         % sets if automatic breaks should only happen at whitespace
    breaklines=true,                 % sets automatic line breaking
    captionpos=b,                    % sets the caption-position to bottom
    commentstyle=\color{dkgreen},      % comment style
    deletekeywords={},            % if you want to delete keywords from the given language
    escapeinside={\%*}{*)},          % if you want to add LaTeX within your code
    extendedchars=true,              % lets you use non-ASCII characters; for 8-bits encodings only, does not work with UTF-8
    frame=none,                      % adds a frame around the code
    keepspaces=true,                 % keeps spaces in text, useful for keeping indentation of code (possibly needs columns=flexible)
    keywordstyle=\bfseries\color{dkblue},       % keyword style
    %identifierstyle=\color{dkyellow},
    language=erlang,                 % the language of the code
    morekeywords={*,...},            % if you want to add more keywords to the set
    numbers=left,                    % where to put the line-numbers; possible values are (none, left, right)
    numbersep=5pt,                   % how far the line-numbers are from the code
    numberstyle=\tiny\color{gray},   % the style that is used for the line-numbers
    rulecolor=\color{black},         % if not set, the frame-color may be changed on line-breaks within not-black text (e.g. comments (green here))
    showspaces=false,                % show spaces everywhere adding particular underscores; it overrides 'showstringspaces'
    showstringspaces=false,          % underline spaces within strings only
    showtabs=false,                  % show tabs within strings adding particular underscores
    stepnumber=1,                    % the step between two line-numbers. If it's 1, each line will be numbered
    stringstyle=\color{blue},       % string literal style
    tabsize=4,                       % sets default tabsize to 2 spaces
    title=\lstname                   % show the filename of files included with \lstinputlisting; also try caption instead of title
}

\renewcommand{\figurename}{Tabell }

\title{Inlämningsuppgift Laboration 5}
\author{Axel Isaksson}
\date{2013-12-11}

\begin{document}

\maketitle 

\section{Uppgiften}

\paragraph\
Denna laborationsuppgift gick ut på att konstruera en enkel webbserver, och
jämföra olika sätt att hantera samtidiga förfrågningar från flera klienter.
I uppgiften ingick att kunna tolka den vanligaste typen av meddelande i
HTTP-protokollet; \emph{GET}-meddelandet, samt att kunna generera ett svar
att skicka tillbaka. Följande metoder jämfördes för att hantera multipla,
samtidiga anslutningar:

\begin{itemize}
    \item Sekvensiell hantering: förfrågningar hanteras en i taget.
    \item Samtidig hantering: varje förfrågan hanteras i en bakgrundsprocess.
    \item Fast antal hanteringsprocesser: förfrågningar tilldelas en
        hanteringsprocess från en fast, delad pott. När inga lediga
        hanteringsprocesser finns köläggs förfrågan.
\end{itemize}

\section{Ansats}

\subsection{HTTP-tolk}

\paragraph\
De första deluppgiften gick ut på att skriva hjälpfunktioner för att tolka
och svara på HTTP-förfrågningar. Då uppgiften endast gick ut på att skapa
en minimal webbserver i övningssyfte hanteras bara \emph{GET}-metoden,
och eventuell metadata i meddelandehuvudet ignoreras. Stora delar av denna
kod var redan given i uppgiften.

\paragraph\
HTTP är ett radbaserat protokoll i klartext (det vill säga ej binärt protokoll)
där varje meddelande består av ett meddelande huvud och en meddelandekropp.
Meddelandehuvudet inleds med en förfrågan (metod) eller ett svar på första raden
och följs av metadata i nyckel/värde-par; ett per rad.
Varje rad avslutas med ett vagnsreturtecken och ett nyradstecken
(förkortat \emph{CRLF}). Meddelandehuvudet skiljs från meddelandekroppen
med en tom rad, bestående av endast \emph{CRLF}.

\paragraph\
HTTP-tolken från första deluppgiften läser ett meddelande rad för rad och
extraherar de olika delarna. Från första raden plockas typen av förfrågan ut.
Implementationen från uppgiften hanterar bara \emph{GET}-förfrågningar som
överensstämmer med HTTP-versionerna 1.0 och 1.1. Ur denna rad extraheras också
sökvägen till den den efterfrågade filen. Följande rader fram till den tomma
raden läses och läggs i en lista med metadata. Resterande rader exkluderat den
tomma tillhör meddelandekroppen. Denna data returneras i en tupel som
\emph{\{Request, Headers, Body\}}. Nedan följer ett exempel på en
HTTP-förfrågan:

\begin{verbatim}
    GET /path/to/file.txt HTTP/1.1
    Metadata-Key: Value
    Another-Key: Value
    
    Message body, may be empty
\end{verbatim}

\subsection{En enkel server}

\paragraph\
Nästa deluppgift gick ut på att implementera en enkel server som kunde ta emot
en HTTP-förfrågan och skicka tillbaka ett svar. Till detta används
\emph{sockets} (löst översatt; ändpunkter) som för programmet är den synliga
delen av en förbindelse till exempel mellan två datorer på internet,
men kan även använas lokalt.
Till detta användes i uppgiften Erlangs \emph{gen\_tcp}-bibliotek.

\paragraph\
Den första iterationen av servern implementerades helt sekvensiellt. En socket
öppnas som lyssnar efter nya anslutningar. När en klient ansluter till servern
accepteras anslutningen och dess förfrågan hanteras. Denna förfrågan behandlas
genom en ny socket som erhålles av accept-funktionen. När klientens förfråga är
behandlad skickas ett svar tillbaka och denna klient-socket stängs. Servern kan
nu acceptera nästa anslutning. Koden för denna deluppgift var också till stora
delar redan given.

\subsection{En testklient}

\paragraph\
I den tredje deluppgiften implementerades en liten testklient för att mäta
serverns prestanda. Klienten skickar ett flertal förfrågningar
(som standard 100) och räknar tiden det tar att få ett svar.
Koden för även denna deluppgift var redan given.

\subsection{Ökad serverkapacitet}

\paragraph\
För att förbättra serverns kapacitet att kunna hantera flera klienter
samtidigt undersöktes i denna deluppgift två olika metoder att
parallellisera hanteringen av förfrågningar.

\paragraph\
Den första metoden gick ut på att låta varje förfrågan hanteras av en egen
process. Då en ny socket öppnas för varje accept kan en process skapas som
hanterar kommunikation genom denna socket; läser förfrågan, svarar, stänger
socketen och avslutar. På detta sätt kan flera samtidiga anslutningar hanteras
utan onödiga väntetider. Nackdelar med denna metod är att det finns ett visst
overhead för att skapa en process, både tismässigt och sett till
systemets minnesanvändning. Detta öppnar också för möjligheten till en
``Denial Of Service''-attack där en anfallare enkelt kan översvämma systemet
med processer som förbrukar systemresurser.

\paragraph\
Den andra metoden gick ut på att, från en pott med ett fast antal
``arbetsprocesser'', tilldela en arbetsprocess till en inkommande förfrågan.
När potten är tom köläggs nya förfrågningar tills en process blir ledig.
I min lösning realiserade jag det genom att använda mig av en LIFO-stack
med arbetsprocesser, som hanteras av en övervakare. Vid en accept
skickas ett meddelande till övervakaren med den tilldelade socketen.
övervakaren tilldelar anslutningen till den första, lediga arbetsprocessen
i stacken, genom att ett meddelande skickas till processen och den tas bort
ur listan med tillgängliga arbetare. När arbetsprocessen har hanterat och
svarat på klientens förfrågan skickar den ett meddelande till övervakaren
där den talar om att den åter är ledig, varpå den läggs på stacken igen.
Anledningen till att jag valde en stack som datastruktur är för att de
stämmer bra överens med erlangs implementation av listor; det går snabbt
att både ta bort och lägga till det första elementet i en lista, och det
spelar ingen roll vilken process som hanterar vilken förfrågan.

\begin{lstlisting}[language=erlang]
pool([]) ->
    receive
        {done, Pid} ->
            pool([{handler, Pid}])
    end;
pool(Handlers) ->
    receive
        {accept, Sock} ->
            [{handler, Pid} | T] = Handlers,
            Pid ! {handle, Sock, self()},
            pool(T);
        {done, Pid} ->
            pool([{handler, Pid} | Handlers])
    end.
\end{lstlisting}

Kod för övervakarfunktionen. Funktionen tar emot \emph{accept}-meddelanden
från huvudprocessen som lyssnar efter nya, inkommande anslutningar och
tilldelar dem till lediga arbetsprocesser. Funktionen tar även emot
\emph{done}-meddelanden från färdiga arbetsprocesser som åter igen ska läggas
på stacken och göras tillgängliga för nya förfrågningar.

\section{Utvärdering}

\paragraph\
Prestandamätning utfördes för de olika servrarna. Resultaten följer i tabell
\ref{tab:results}. All prestandamätning skedde med en artificiell väntetid
på $10 ms$ för att simulera filsystemåtkomst och skriptning hos en riktig
webbserver. I mätningen kördes server och klient på två olika, fysiska
maskiner, varför en okänd och variabel väntetid för att skicka paketen över
nätverket kan ha påverkat resultatet. Numren är normaliserade över flera
körningar.

\begin{table}
\centering
\begin{tabular}{|l|r|r|r|r|}
\hline
                & Sekvensiell   & Samtidig  & \multicolumn{2}{|c|}{Arbetsprocesser}\\
\hline
Processer       & 1             & $\infty$      & 10            & 50            \\
Förfrågningar   &               &               &               &               \\
\hline
1               & 4502 $\mu$s  & 4459 $\mu$s   & 4052 $\mu$s   & 3875 $\mu$s   \\
\hline
10              & 8707 $\mu$s  & 8953 $\mu$s   & 11181 $\mu$s  & 15986 $\mu$s  \\
\hline
50              & 30081 $\mu$s  & 34193 $\mu$s  & 32197 $\mu$s  & 35127 $\mu$s  \\
\hline
100             & 106898 $\mu$s & 58938 $\mu$s  & 63075 $\mu$s  & 63312 $\mu$s  \\
\hline
\end{tabular}
\caption{Prestandamätning av de olika servrarna}
\label{tab:results}
\end{table}

\paragraph\
Från tabellen kan utläsas att för få antal samtida förfrågningar presterar de
olika servrarna relativt lika. För högre belastning presterar servrarna
med flera processer något bättre än den sekvensiella servern, men resultaten är
förvånansvärt lika, något som kan ha att göra med hur erlang hanterar sina
processer. Värt att notera är också att för de testkörningar jag gjorde
med server och klient i samma erlangskal (resultat ej med i tabell)
tog körningar med endast tio samtidiga förfrågningar mellan en och två
\emph{sekunder} att slutföra, för samtliga servrar.

\section{Sammanfattning}

\paragraph\
Sammanfattningsvis kan sägas att resultatet av försöken var något oväntat;
jag hade förväntat mig större skilnad i prestanda mellan den sekvensiella
och de parallella servrarna för flera samtidiga anslutningar.
Prestandavinst och andra fördelar med en flerprocessig lösning
går dock naturligtvis fortfarande att finna för verkliga
användningsområden, där servern faktiskt läser stora filer från filsystmet,
eller kör tunga skript för vissa förfrågningar. I dessa fall ska inte servern
låsa sig med endast den tidskrävande förfrågningen, och låta alla andra
klienter vänta tills dess att den är klar.

\paragraph\
Angående uppgiften så tycker jag att den var lärorik och ganska rolig, fast
jag även denna gång hoppats på lite mindre färdig kod i uppgiften.
Inga större problem stöttes heller på under arbetet.

\end{document}
