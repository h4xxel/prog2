%% En enkel mall för att skapa en labb-raport.
\documentclass[a4paper, 11pt]{article}
\usepackage[utf8]{inputenc} 
\usepackage[swedish]{babel}
\usepackage{verbatim}
\usepackage{hyperref}

\renewcommand{\figurename}{Tabell }

\title{Inlämningsuppgift Laboration 4}
\author{Axel Isaksson}
\date{2013-12-04} 

\begin{document}

\maketitle 

\section{Uppgiften}

\paragraph\
Laborationsuppgiften gick ut på, att som en grundläggande övning i
distribuerade system, simulera en modell med ätande filosofer vid ett runt
bord. Mellan varje par filosofer ligger \emph{en} ätpinne, och varje filosof
måste ha två pinnar för att kunna äta. Då filosoferna, utöver att äta, även
sitter och filosoferar, kan två filosofer dela på ätpinnen mellan sig.

\paragraph\
De senare deluppgifterna gick ut på att lösa och diskutera olika problem som
uppstod, bland annat när hela systemet låste sig på grund av att varje filosof
kunde plocka upp en ätpinne och sedan i oändlighet vänta på den andra.

\section{Ansats}

\paragraph\
De första tre deluppgifterna gick ut på att implementera det grundläggande
systemet med fem filosofer och fem ätpinnar. Varje ätpinne och varje filosof
skulle köra som en egen process där varje ätpinne håller sitt eget tillstånd
(ledig eller upptagen). Filosoferna plockar upp och lägger tillbaka pinnarna
genom att meddelanden skickas mellan processerna så att pinnarna byter
tillstånd. Genom att på detta sätt använda Erlangs processhantering
blir systemet lätt att implementera, och filosoferna kommer automatiskt att
vänta på att en ätpinne blir ledig genom Erlangs meddelandekö.

\paragraph\
Följande uppgifter gick ut på att experimentera med systemet för att göra det
mer stabilt, samt minimera risken för stillestånd. Systemet i den första
versionen initierade med fem lediga ätpinnar och fem ``tänkande'' filosofer.
När alla dessa filosofer tänkt klart plockade de alla upp sin vänstra
ätpinne, vilket ledde till stillestånd då alla väntade på att få sin högra
ätpinne ledig.

\paragraph\
Den första, och något naiva, lösningen till detta problem var att introducera
ett slumpelement: längen på varje filosofs tänk- och ätcykel baseras på ett
slumpvärde, vilket minskar risken för att filosoferna låser varandra.
Anledningen till att lösningen är naiv är att den inte egentligen löser
problemet, utan istället försöker gå runt det och ``hoppas'' att det fungerar.
I princip har bara mer indeterminism tillförts vilket kör att vi inte längre
kan säga när systemet kommer fungera eller inte, vilket är långt från önskvärt.

\paragraph\
Nästa steg var att lägga till en timeout när filosoferna väntar på ätpinnarna.
En filosof som väntat för länge på en ätpinne ger upp, lägger ner sin andra
ätpinne om han höll i någon, och återgår till att tänka ett tag till.
Problemet med denna lösning är, att utan ett slumpelement så kommer denna
lösning fungera på precis samma sätt som den första. Om filosoferna sträcker
sig efter sin vänstra ätpinne samtidigt, och sedan väntar precis lika länge
på att sin högra pinne ska bli ledig innan de ger upp och lägger ner
vänsterpinnen, kommer systemet fortfarande att låsa sig. I kombination med
slumpvärden för väntetider och timeouts fungerar dock systemet utan att låsa
sig, men med minskat genomflöde på grund av onödiga väntetider. Det gäller
dock att vara aktsam på att förfrågan om att plocka upp ätpinnen från en
filosof som når timeout och ger upp fortfarande ligger kvar i pinnens
meddelandekö, och måste rensas ut eller ignoreras för att inte orsaka
senare problem genom att pinnen låser sig till en filosof som inte längre
är intresserad av den.

\paragraph\
Det tredje steget för att förbättra systemet var att använda
asynkron kommunikation mellan filosoferna och ätpinnarna. Detta innebär att
filosoferna skickar iväg meddelanden för att plocka upp båda ätpinnarna
``samtidigt'' och sedan väntar tills dess att pinnarna blir lediga, i stället
för att först plocka upp den ena och sedan den andra. På detta sätt minskas
onödiga väntetider, och mer dynamik införs i systemet som medför bättre
konflikthantering då filosoferna kan plocka upp vilken som av pinnarna när
de blir lediga.

\paragraph\
Ett centraliserat system presenterades sedan där en ``servitör'' ansvarar för
att dela ut pinnar till de ätande filosoferna.
Då varje filosof behöver exakt två pinnar för att kunna äta innebär min lösning
att ätpinnarna delas ut i par av servitören. Filosoferna, när de vill
börja äta, ställer sig i kö hos servitören i väntan på ett par ätpinnar.
Servitören delar ut sina två par ätpinnar till de första i kön, och delar sedan
flygande ut ätpinnarna parvis till nästvidkommande i kön så fort ett par lämnats
tillbaka från en filosof som ätit klart. Så länge filosoferna lämnar tillbaka
sina ätpinnar när de är klara låser sig inte systemet. En kö bildas dock med
filosofer som bara sitter och väntar på att ett par ätpinnar ska bli
tillgängliga, vilket tar tid från deras filosoferande, även om användandet av
pinnarna optimeras.

\section{Utvärdering}

\paragraph\
För ett litet system med få noder, som dessutom inte är tidskritiskt kan
ett system som baserar sig på slumpmässiga timeouts och väntetider fungera.
Som exempel kan tas ALOHAnet\footnotemark[1] som använder slumpmässiga
väntetider för att hantera krockar.
När det dock gäller större system med fler noder som medför fler krockar,
eller i system där högre datagenomströmning krävs så att artificiella
väntetider är oacceptabelt måste en annan lösning finnas.

\paragraph\
Man skulle kunna tänka sig flera lösningar på detta problem. I uppgiften
föreslogs en centraliserad lösning med en ``servitör'' som kan på begäran av
en filosof dela ut ett par ätpinnar. En annan lösning är en decentraliserad
modell där filosoferna runt bordet kan fråga varandra om de får låna ätpinnar.
I denna modell måste filosoferna dels kunna ``hitta'' varandra och sedan genom
kommunikation sinsemellan komma överens om vem som ska få använda ätpinnarna
och när. Ett sådant system är svårare att designa, men lämpar sig väl
för applikationer i större skala; systemet ska kunna fortsätta att fungera
oberoende av den centrala noden (här: servitören) ifall den skulle gå ner.

\paragraph\
En mer praktisk och lätthanterlig lösning är en hybrid där noderna använder
en central nod för att finna varandra, men kan sedan kommunicera sinsemellan.
Systemet kan fungera flytande, utan central nod, kortare stunder om denna
skulle gå ner, men noderna måste då och då skynkronisera sin ``adressbok''
med den centrala servern för att till exempel hitta nya noder. Man skulle
även kunna tänka sig ett nätverk av grannar. I filosofuppgiften skulle det
innebära att en filosof som inte har tillgång till sina ätpinnar kan fråga
sina bordsgrannar, och låta förfrågan vandra runt från nod till nod tills
dess att man finner en ledig ätpinne som kan skickas tillbaka. Nackdelarna
med ett sådant system är bland annat (för att fortsätta med filosof-analogin)
att en filosof som plötsligt lämnar bordet bryter kedjan, samt att filosoferna,
som varken vill bli störda när de tänker eller prata med mat i munnen, inte kan
vidarebefodra ett meddelande förns de är klara. För att öka både redundans
och kapacitet måste varje nod ha fler grannoder i nätverket, så att det finns
alternativa vägar mellan noder och så nätverket kan reparera sig själv om en
nod tappar anslutningen.

\footnotetext[1]{ALOHAnet - Wikipedia, the free encyclopedia
    \url{https://en.wikipedia.org/wiki/ALOHAnet}}

\section{Sammanfattning}

\paragraph\
För att sammanfatta den här uppgiften kan sägas att jag fann den konceptuellt
mycket intressant, och även om det inte var så mycket kod att skriva, så
lämnade uppgiften rum för mycket eget experimenterande och diskuterande.

\paragraph\
Ett par mindre problem uppstod längs vägen, till exempel att varje process
initieras med samma slumpfrö, vilket ledde till att slumporganiseringen
av processerna inte fungerade. Detta löstes dock enkelt genom att låta varje
filosofprocess initiera sin slumpgenerator med hjälp av Erlangs inbyggda
\emph{now}-funktion som garanterar ett unikt värde varje anrop.

\end{document}
