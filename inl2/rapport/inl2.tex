%% En enkel mall för att skapa en labb-raport.
\documentclass[a4paper, 11pt]{article}
\usepackage[utf8]{inputenc} 
\usepackage[swedish]{babel}
\usepackage{verbatim}

\renewcommand{\figurename}{Tabell }

\title{Inlämningsuppgift Laboration 2}
\author{Axel Isaksson}
\date{2013-11-20} 

\begin{document}

\maketitle 

\section{Uppgiften}

\paragraph{Huffmankodning}
Uppgiften gick ut på att implementera en algoritm; huffmankodning; som är
en enkel komprimeringsalgoritm lämpad för vanlig text. Komprimeringen
går ut på att komprimera olika tecken (bokstäver) med olika längd som beror
på hur ofta de förekommer i texten. Detta görs utan att en speciell separator,
eller prefix måste infogas som avskiljare mellan varje tecken, vilket gör att
varje bitsträng som representerar ett tecken kan packas direkt inpå föregående.

\paragraph\
I uppgiften ingick också att mäta prestandan på implementationen för stora
texter.

\section{Ansatts}

\paragraph\
Huffmankodning använder sig av ett träd för att representera ett teckens
bitsträng. I detta träd ligger varje tecken som ett löv, där ofta förekommande
tecken ligger närmare roten, och vägen genom trädet till lövet motsvarar bitar
i strängen; i min lösning betyder vänstergren en binär nolla och högergren en
binär etta.

\paragraph\
Detta träd byggs upp utifrån en exempeltext, för vilken en frekvenstabell över
förekommande tecken skapas. Denna frekvenstabell ordnas som en prioritetskö
där tecknen med lägst frekvens prioriteras längst fram i kön.
Huffmanträdet skapas sedan genom att extrahera de två första elementen från
kön, skapa en nod i trädet vars barn är de två elementen, sätta nodens
"prioritet" till summan av de två barnens. Denna nod placeras sedan på rätt
ställe i prioritetskön. Processen upprepas tills endast ett element återstår:
rotnoden.

\paragraph\
När huffmanträdet är skapat gås det igenom, och en tabell upprättas över alla
tecken och deras bitsträngar (som ges av tecknets sökväg i trädet, enligt
ovan). Kodning (komprimering) av text sker genom att texten gås igenom,
tecken för tecken, och varje tecken ersätts av sin respektive bitstäng
(tätt packad på föregående). Avkodning (dekomprimering) sker enklast genom
att använda huffmanträdet, och genom att för varje bit på indatan gå till
antingen höger eller vänster i trädet. När ett löv nås kan bitsträngen
ersättas med ett tecken, och processen startar om med nästa bit från
trädets rot.

\paragraph\
I min implementation använder jag ett binärt sökträd för att representera
frekvenstabellen. Trädet är först sorterat på nyckel (bokstavens/tecknets
motsvarighet i ASCII). Detta gör att det går (relativt) fort att söka
igenom exempeltexten och bygga tabellen. När texten är genomsökt sorteras
trädet om efter värde (frekvens) istället för nyckel. Detta gör att trädet
nu enkelt kan användas som en prioritetskö, där elementet med lägst frekvens
(högst prioritet) finns längst ner till vänster.

\paragraph\
Jag bygger nu huffmanträdet genom att extrahera de två elementen med lägst
frekvens enligt förklaring ovan. För att kunna hålla koll på trädet som
konstrueras, och för att kunna lägga tillbaka de skapade föräldernoderna
i prioritetskön, utan att deras barnnoder som redan är förbrukade ska
kunna plockas ut igen (och algoritmen fastna i en oändlig loop), använder
jag mig av nyckelelementet (som är ett tecken för alla lövnoder, men som
är helt oanvänt för de nyskapade föräldernoderna) till att hålla en tuppel
med det nya huffmanträdet. Detta kan tyckas inte vara den snyggaste lösningen,
men den fungerar.

\paragraph\
\begin{verbatim}
{node, Key, Value, Left, Right}
\end{verbatim}
Strukturen för en nod i trädet. Key är tecknet i ASCII, Value är frekvensen.
När ett element extraheras från trädets yttersta vänstergren i prioritetskön
ersätts det av sin högergren.
De två lägst prioriterade noderna kopplas ihop genom en föräldernod med
följande struktur:
\begin{verbatim}
{node,
    {
        {huffman, Key1, Val1}, 
        {huffman, Key2, Val2}
    },
    Val1 + Val2, null, null
}
\end{verbatim}

Detta nya element integreras sedan med trädet igen genom att placeras på rätt
ställe i kön.

\paragraph\
När processen är klar återfinns det färdiga huffmanträdet i köns nu enda
kvarvarande elements nyckelfält. Detta träd gås igenom inorder
(vänstergrenen behandlas innan aktuella noden som behandlas innan högergrenen)
och en kodningstabell upprättas genom att varje tecken tilldelas en bitsträng
baserad på dess sökväg, enligt ovan. Kodningsfunktionen använder denna tabell
för att översätta varje tecken i indatatexten till en bitsträng. Tabellen
ligger i min implementation i en lista, vilket inte är helt optimalt.
Helt optimalt hade varit en array för konstant uppslagningshastighet. Näst bäst
(och som dessutom går att implementera i erlang) hade varit att använda ett
binärt sökträd, på samma sätt som jag gjorde för söktabellen.

\paragraph\
Gällande avkodningsfunktionen implementerade jag aldrig en avkodningstabell som
uppgiften föreslog. Detta då huffmanträdet direkt fungerar utmärkt för
avkodning; indata-bitsträngen avgör sökvägen ner i trädet tills ett löv nås,
vilket motsvarar ett tecken i ASCII.

\paragraph\
För prestandamätning använde jag följande funktion från labb 1, som jag
anpassat till ändamålet:

\begin{verbatim}
time(Fun)->
    %% time in micro seconds
    T1 = now(),
    Res = Fun(),
    T2 = now(),
    {timer, timer:now_diff(T2, T1), Res}.
\end{verbatim}

Funktionen returnerar en tuppel med både returvärdet från den
benchmarkade funktionen, samt exekveringstiden.

\section{Utvärdering}

\paragraph\
Förutom de korta exempeltexterna från uppgiften testade jag min huffman\-
implementation med två störe texter: en genererad \emph{Lorem Ipsum}-
text på 34 kB samt en bok -- Victor Hugos Les Misérables, engelsk översättning
i textformat från Projekt Gutenberg på dryga 3 MB. Resultaten kan ses i
tabellerna för komprimeringsgrad [Tabell \ref{tab:komprimering}] och
exekveringstid [Tabell \ref{tab:exekvering}]. I båda försöken användes
texterna själva som exempeltexter för tabellgenereringen, vilket medför
optimal komprimering.

\paragraph\
Ett par försök med riktigt litet alfabet gjordes också. Då ett litet alfabet
redan kan representeras av ett fåtal bitar är effekten av komprimeringen
mindre -- i vissa fall försummbar.

\begin{table}
\centering
\begin{tabular}{|l|r|r|r|}
\hline
    Text            & Storlek (Byte)& Komprimerad (Byte)& Komprimeringsgrad \\\hline
    Lorem Ipsum     & 33919         & 18012             & 53\%              \\\hline
    Les Misérables  & 3240340       & 1823456           & 56\%              \\\hline
\end{tabular}
\caption{Storlek och komprimeringsgrad för de olika texterna}
\label{tab:komprimering}
\end{table}

\begin{table}
\centering
\begin{tabular}{|l|r|r|r|}
\hline
    Text            & Initiering ($\mu s$)  & Komprimering ($\mu s$)& Dekomprimering ($\mu s$)  \\\hline
    Lorem Ipsum     & 16784                 & 6059                  & 4133                      \\\hline
    Les Misérables  & 775279                & 1322659               & 353677                    \\\hline
\end{tabular}
\caption{Exekveringstider för de olika stegen för de olika texterna}
\label{tab:exekvering}
\end{table}

\paragraph\
Av datan följer att en $100 \times$ ökning i textstorlek medför en $50 \times$
ökning av exekveringstiden för initieringsteget (generering av tabeller och
huffmanträd), en $200 \times$ ökning av exekveringstiden för komprimering
och en $100 \times$ ökning av exekveringstiden för dekomprimering.

\paragraph\
Koden har uppenbart rum för förbättringar. Som redan nämnt kan komprimerings\-
funktionen skrivas om till att använda en effektivare datastruktur än en lista
för att representera omkodningstabellen. Då erlang inte har arrayer på samma
sätt som till exempel C kan kanske ett träd användas för viss uppsnabbning.
Erlangs proplists kan också vara ett alternativ som fungerar som uppslagnings\-
tabell från nyckel till värde.

\section{Sammanfattning}

Sammanfattningsvis kag jag säga att även denna övning gick bra, även om en
del tid och tankekraft behövde läggas ner för att få alla tabeller och träd
att sitta ihop och bete sig som förväntat. Jag stötte på ett par problem där
mina träd var vända åt fel håll, och även bitsträngarna som representerar
bokstäver. Dessa problem kunde dock lösas relativt enkelt, med hjälp av
spårutskrifter och erlangskalet, men det märks tydligt att det är lätt att
något blir fel när det är många olika delar som ska passa ihop och språket
ännu är ganska främmande.

\end{document}
