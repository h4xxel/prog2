%% En enkel mall för att skapa en labb-raport.
\documentclass[a4paper, 11pt]{article}
\usepackage[utf8]{inputenc} 
\usepackage[swedish]{babel}
\usepackage{verbatim}

\renewcommand{\figurename}{Tabell }

\title{Inlämningsuppgift Laboration 1}
\author{Axel Isaksson}
\date{2013-11-13} 

\begin{document}
\maketitle 

\section{Uppgiften}

Uppgiften bestod av tre deluppgifter:
\begin{itemize}
    \item Vända på en lista
    \item Omvandla ett heltal till en lista med binära bitar
    \item Räkna fibonacci-tal
\end{itemize}

\paragraph\
Den första uppgiften gick ut på att på två olika sätt vända på en lista.
(Det första elementet skulle hamna sist och det sista först.)
Detta skulle göras på två olika sätt; genom att succesivt ta bort det första
elementet, anropa sig själv för att vända på resten av listan, och sedan
lägga till det borttagna elementet sist till den vända listan, och genom att
använda sig av en ackumulatorlista där man som med en kortlek tar det
översta elementet och lägger på toppen av ackumulatorlistan som sedan blir
den nya, vända listan. Dessa två metoder skulle jämföras i snabbhet genom
en färdigskriven benchmark-funktion.

\paragraph\
Den andra uppgiften gick ut på att, på två valfria sätt, omvanda ett heltal
till en lista med binära bitar. Talet $369_{10} = 101110001_{2}$ skulle
alltså generera listan \emph{[1, 0, 1, 1, 1,0, 0, 0, 1]}. Fördelar/nackdelar
med de olika metoderna skulle diskuteras.

\paragraph\
Den tredje uppgiften gick ut på att skriva en funktion som ränknar fram
fibonaccinummer och mäta funktionens prestanda genom att anpassa benchmark\-
funktionen från första uppgiften. 

\section{Ansats}

\subsection{Första uppgiften}

\paragraph\
Koden till dena uppgift var i stort sett redan given i uppgiften. De två
funktionerna \emph{nreverse} och \emph{reverse} skulle jämföras
prestandamässigt.

\paragraph{nreverse}
\begin{verbatim}
nreverse([]) -> [];
nreverse([H|T]) -> nreverse(T) ++ [H].
\end{verbatim}

Funktionen \emph{nreverse}, en naiv approach till att vända en lista. Den vända
listan erhålles genom att det första elementet, \emph H, tas bort från
listan och läggs till på slutet av den vända lista som erhålles genom att
rekursivt anropa sig själv med resten av listan, utan det första elementet.

\paragraph{reverse}
\begin{verbatim}
reverse(L) -> reverse(L, []).
reverse([], R) -> R;
reverse([H|T], R) -> reverse(T, [H|R]).
\end{verbatim}

Funktionen \emph{reverse}, en bättre lösning på problemet. Här används listan
\emph R som en ackumulator som skickas med i varje funktionsanrop och som
växer och bildar den nya, omvända listan.
Funktionen kan liknas vid att ta ett kort från toppen av en kortlek och placera
i en ny hög. När alla kort dragits från toppen av ursprungshögen och placerats
på toppen av ackumulatorhögen kommer ordningen i ackumulatorn vara omvänd
jämtemot ursprungshögen.
Då funktionen måste ta två argument, en arbetslista och en ackumulatorlista,
finns också en wrapperfunktion som bara tar ett argument för att förenkla
användning och göra funktionen kompatibel med nreverse.

\paragraph\
Benchmarkfunktionen fungerar på så sätt att den skapar ett flertal listor
av ökande längd, anropar funktionerna reverse och nreverse på varje lista,
och mäter tiden det tar för funktionen att bli klar.

\subsection{Andra uppgiften}

\paragraph\
I denna uppgift skulle man på två olika, valfria sätt konvertera ett heltal
till en lista med binära bitar (ex \emph{[1, 0, 0, 1, 1]}).

\paragraph{Bitshift}
\begin{verbatim}
unpack1(Bin) -> unpack1(Bin, []).
unpack1(0, []) -> [0];
unpack1(0, Acc) -> Acc;
unpack1(Bin, Acc) -> unpack1(Bin bsr 1, [Bin band 1 | Acc]).
\end{verbatim}

Då jag programmerat mycket i C tidigare var denna lösningen den direkt
uppenbara för mig. Talet shiftas ner, en bit i taget, och logiskt AND utförs
för att maska ut den lägsta biten, som sedan läggs till ackumulatorlistan
\emph{Acc}. Jag var tungen att lägga till matchning mot talet 0 med tom
ackumulator för att en lista med en nolla (\emph{[0]}) skulle returneras
istället för en tom lista.

\paragraph{Inbyggda funktioner}
\begin{verbatim}
unpack2(Bin) ->
    lists:map(fun(C) -> C - 48 end, integer_to_list(Bin, 2)).
\end{verbatim}

I min andra lösning använder jag mig av den inbyggda funktionen
\emph{integer\_to\_list} som gör om ett heltal till en lista i en given bas.
Här användes basen två, vilket returnerar en lista med utskrivbara tecken,
som kan representeras som en \emph{sträng} likt "10010101", vilket inte
riktigt vad det som söktes, varför följande lösning på problemet konstruerades:
\emph{lists:map} utför en funktion på varje element i listan; i detta fall
en annonym funktion som konverterar tecknet "n"
(ASCII 48 = '0', ASCII 49 = '1') till heltalet n.
Således: $'1' - 48 = 1$.

\subsection{Tredje uppgiften}

\paragraph\
Denna uppgift gick ut på att skriva en funktion som räknar fram fibonacci\-
nummer (de två första nummren är 0, 1, varje övrigt nummer kan konstrueras
genom att addera ihop de två tidigare). Jag valde att göra två olika funktioner
som räknar fram numren på lite olika sätt, för att jämföra dem mot varandra.
Den första är rekursiv och väldigt naiv. För varje tal som ska räknas ut
anropar den sig själv två gånger, för att hitta de två föregående talen i
serien och adderar sedan ihop dem. Den andra funktionen använder en
iterativ approach, även om jag här implementerade den som en rekursiv loop.

\paragraph{Iterativ fibonacci}
\begin{verbatim}
fib1(N) -> fib1(0, 1, N).
fib1(Fib1, _, 0) -> Fib1;
fib1(Fib1, Fib2, N) -> fib1(Fib2, Fib1 + Fib2, N - 1).
\end{verbatim}

Denna funktion anropas med de två starttalen \emph 0 och \emph 1 och räknar
fibonaccitalen uppåt samtidigt som \emph N används som räknare som räknas
ner till 0 och det sökta talet i serien har hittats.
På detta sätt behöver varje tal bara räknas ut en gång, till skilnad mot den
förra lösningen.

\section{Utvärdering}

\subsection{Första uppgiften}

\begin{table}
\centering
\begin{tabular}{|l|r|r|}  
\hline
Listlängd & Tid nreverse ($\mu s$) & Tid reverse ($\mu s$)\\\hline
    16  & 225          & 48   \\\hline
    32  & 662	       & 73   \\\hline
    64  & 2591	       & 98   \\\hline
    128 & 11014        & 246  \\\hline
    256 & 18624        & 164  \\\hline
    512 & 65010        & 273  \\\hline

\end{tabular}
\caption{Exekveringstid för funktionerna 
    \emph{nreverse} och \emph{reverse}}
\label{tab:uppg1}
\end{table}

\paragraph\
I tabellen \ref{tab:uppg1} kan vi se att funktionen \emph{reverse} är
betydligt snabbare än den naivare men mer straight-forward \emph{nreverse}.
Detta beror på ett flertal saker:

\begin{itemize}
    \item \emph{nreverse} skapar i ihopslagningen i varje steg en kopia av hela
        listan som sedan ändå tas bort på en gång i och med append (++).
        \emph{reverse} använder istället en ackumulator som skickas med i varje
        anrop och som i varje steg används som svans vid skapandet av den nya
        ackumulatorlistan vilken bara får en ny första post med cons-operatorn.
        Detta har konstant tidskomplexitet.
    \item \emph{reverse} är svansrekursiv; den gör inget mer med sitt
        returvärde än att returnera det vilket leder till bättre optimering,
        mindre minnesanvändning och därför mindre overhead än \emph{nreverse}.
\end{itemize}

\subsection{Andra uppgiften}

\paragraph\
I denna uppgift väljer jag att kalla min första lösning för ``den bättre''
av dem. Kanske för att jag är van vid det sättet att manipulera bitar, men
också då funktionen bara behöver gå igenom varje bit en gång och gör allt i
en enda loop. Även om det är tillfresställande med enradslösningar som min
andra alternativa lösning är den varken särskilt snygg eller särskilt
effektiv då den först måste konvertera numret till en sträng av 1:or och 0:or
och sedan gå igenom igen och konvertera varje tecken i strängen till ett tal.

\subsection{Tredje uppgiften}

\paragraph\
När jag kör benchmark för mina två olika implementationer av fibonicci\-
räknaren är det direkt väldigt tydligt att den andra, iterativa varianten
är direkt överlägsen.
Min snabbare implementation av fibonacciräknaren räknar ut det 500000:e talet
på ungefär hälften av tiden som det tar den långsammare implementationen att
räkna ut det 40:e. Anledningen till att det är så är lätt att se om man tittar
i koden: den långsamma implementationen med rekursiv approach börjar uppifrån
och anropar sig själv två gånger för att ta reda på de två tidigare talen i
serien; dessa anrop anropar i sin tur sig själv ytterligare två gånger vardera
för att hitta sina två tidigare tal, och tal som man redan räknat ut
``återanvänds'' inte utan räknas ut igen när de behövs. Detta leder till
kvadratisk tidskomplexitet, vilket framgår i tabell \ref{tab:uppg3}.

\paragraph\
Den snabbare implementationen börjar istället från nedifrån och räknar ut
talen tills den når det sökta, vilket ger den linjär tidskomplexitet.
En funktion för att estimera tidsåtgång kan approximeras till
${8n\over{5}} \times{10^{-6}} s$.

\begin{table}
\centering
\begin{tabular}{|l|r||l|r|}  
\hline
Fib\# & Tid naiv fibonacci ($\mu s$) & Fib\# & Tid snabb fibonacci ($\mu s$)\\
\hline
    20  & 8297          & 2000  & 2690 \\\hline
    22  & 10021	        & 2200  & 1763 \\\hline
    24  & 26792	        & 2400  & 1714 \\\hline
    26  & 68669         & 2600  & 1892 \\\hline
    28  & 179195        & 2800  & 2158 \\\hline
    30  & 469932        & 3000  & 2458 \\\hline
    32  & 1229740       & 3200  & 2732 \\\hline
    34  & 3217523       & 3400  & 3004 \\\hline
    36  & 8454301       & 3600  & 3316 \\\hline
    38  & 22166482      & 3800  & 3677 \\\hline
    40  & 57957391      & 4000  & 4065 \\\hline

\end{tabular}
\caption{Exekveringstid för olika fibonacciräknare}
\label{tab:uppg3}
\end{table}

\section{Sammanfattning}

\paragraph\
Sammanfattningsvis tycker jag att uppgifterna gått bra och utan allt för mycket
krångel. Resultaten som erhölls var vad som kunde förväntas och de mindre
avvikelserna som visas i tabellerna ovan kan förklaras med att det behövs
funktionsoverhead som gör sig extra tydlig för små tal.
Det svåraste var att komma över tröskeln och börja tänka funktionellt,
då jag är van vid imperativa språk som C. Det verkar dock finnas en hel del
dokumentation om erlang som hjälper en att komma igång och erlang-skalet
är väldigt användbart för att snabbt kunna testa olika funktioner och
nya idéer.

\end{document}
