%% En enkel mall för att skapa en labb-raport.
\documentclass[a4paper, 11pt]{article}
\usepackage[utf8]{inputenc} 
\usepackage[swedish]{babel}

\title{Rapportens namn}
\author{Ditt Namn}
\date{Datum} 

\begin{document}

\maketitle 

\section{Uppgiften}

Vad går uppgiften ut på, beskriv kortfattat problemet och hur det skulle lösas.

\section{Ansatts}

Om du vill ta med kod för att illustrera ett exempel så kan du göra det så här:

\begin{verbatim}
this(X) ->
    Y = is(X),
    a(test(Y)).
\end{verbatim}

Var noga med att inte använda tabbar i koden, dessa kommer endast att
tolkas som ett mellanslag och indenteringen kommer att bli helt fel. 

\section{Utvärdering}

Uppgifterna i denna kurs kanske inte kräver en utvärdering men man kan
här visa resultat från testkörningar mm. Om man vill lägga in resultat
från testkörningar kan man sammanfatta dessa i en tabell. Ett exempel
kan ses i tabell \ref{tab:results}. 


\begin{table}
\centering
\begin{tabular}{|l|r|r|}  
\hline
kärnor & tid & uppsnabbning\\
\hline
1 & 400ms & 1\\
\hline
2 & 240ms & 1.7\\
\hline
4 & 140ms & 2.8\\
\hline
\end{tabular}
\caption{Ha alltid en rad som förklarar vad tabellen handlar om.}
\label{tab:results}
\end{table}

Om ni vill ha med en graf så rekommenderas ni att använda gnuplot. Om
ni lär er använda det för att göra enkla diagram så kommer ni ha
mycket nytta av det i framtiden.


\section{Sammanfattning}

Vad gick bra och vad gick mindre bra? Vad var de största problemen och
hur löstes de? Skriv en kort sammanfattning.

\end{document}
