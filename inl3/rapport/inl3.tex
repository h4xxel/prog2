%% En enkel mall för att skapa en labb-raport.
\documentclass[a4paper, 11pt]{article}
\usepackage[utf8]{inputenc} 
\usepackage[swedish]{babel}
\usepackage{verbatim}

\renewcommand{\figurename}{Tabell }
\textwidth 5.2in

\title{Inlämningsuppgift Laboration 3}
\author{Axel Isaksson}
\date{2013-11-27} 

\begin{document}

\maketitle 

\section{Uppgiften}

\paragraph\
Denna laborationsuppgift gick ut på att växla tågvagnar.
Ett tåg, bestående av flera (unika) vagnar, skulle kopplas om i en annan
vagns\-följd genom att växlas mellan tre olika spår i en spårväxel.
Uppgiften var indelad i sju deluppgifter, varav den sista var frivillig.
I de första tre uppgifterna implementerades hjälpfunktioner för att manipulera
listor och flytta vagnar mellan olika spår. Övriga uppgifter gick ut på att,
givet ett ursprungligt tåg och ett färdigt tåg, utföra en serie 
vagnsförflyttningar mellan spåren för att från ursprungståget bygga det färdiga
tåget.
Det ingick också i de senare uppgifterna att optimera sin funktion, så att det
färdiga tåget kunde skapas med ett färre antal drag.

\paragraph\
Spåret i uppgiften är tredelat; det har ett huvudspår där det ursprungliga
och det färdiga tåget ska stå, samt två hjälpspår som används som temporära
uppställningsplatser för vagnar som växlas och byter plats. En vagn kan bara
växlas mellan två hjälpspår genom att först växlas via huvudspåret.

\section{Ansats}

\subsection{Listmanipulering}
\paragraph\
Denna delupggift gick ut på att skriva ett par hjälpprocesser för att
dela en lista vid en viss punkt (och returnera antingen elementen före eller
efter punkten), slå samman listor samt att i en lista hitta ett specifikt
element. Detta kändes en aning redundant, då flertalet av dessa funktioner
redan finns i erlangs lists-bibliotek.

\subsection{Vagnsflytt}
\paragraph\
I denna del skulle en funktion för att flytta ett kolli bestående av en
eller flera (ihopkopplade) vagnar mellan huvudspåret och ett av de övriga
spåren skapas. Funktionen skulle som argument ta en förflyttningsinstruktion
och ett ``tillstånd'' bestående av en bangård med vagnarna på de tre spåren
i form av listor. Denna funktion skulle utföra förflyttningsinstruktionen
(i stil med ``flytta två vagnar från huvudspåret till sidospår 1'') på
bangården och returnera ett nytt tillstånd.
Specifikt skulle förflyttningsinstruktionen vara en tupel bestående av en
atom för ett av de två sidospåren, samt ett heltal för hur många vagnar som
skulle flyttas. Talets polaritet skulle bestämma förflyttningens riktning.

\paragraph\
Funktionen implementerades med hjälpfunktionerna från deluppgift 1.
En förflyttning från huvudspåret tar bort \emph N vagnar från slutet av
huvudspårets tåg och placerar dem i början av tåget på det valda hjälp\-
spåret. På samma sätt medför alltså en förflyttning till huvudspåret att
\emph N vagnar tar från början av hjälpspårets tåg och kopplas på slutet
av tåget på huvudspåret.

\subsection{Serieförflyttningar}
\paragraph\
Här gick uppgiften ut på att, med hjälp av flyttfunktionen från uppgift 2,
utföra på ett initialt tillstånd en serie förflyttningar och returnera
en lista med alla tillstånd som gås igenom under processen. Det första
elementet i listan skulle vara det initiala tillståndet innan förflyttningarna,
och det sista elementet det slutgiltiga.

\paragraph\
Denna funktion kunde enkelt implementeras med rekursion på följande sätt:

\begin{verbatim}
move([], State) ->
    [State];
move([Move | Moves], State) ->
    [State | move(Moves, single(Move, State))].
\end{verbatim}

\subsection{Hitta lösningar}
\paragraph\
Denna uppgift gick ut på att implementera funktionen \emph{find} som,
givet ett ursprungligt tåg och ett slutgiltigt tåg, utför en serie vagns\-
förflyttningar för att transformera det ursprungliga tåget till det
slutgiltiga. Funktionens returvärde är en lista med förflyttningar som
utför transformationen.

\paragraph\
I uppgiften presenterades en rekursiv algoritm som skulle användas:
Den första vagnen i det slutgiltiga tåget används som referens.
Ursprungståget (som står på huvudspåret) söks igenom efter denna vagn
och delas upp på denna position. Alla vagnar efter denna vagn samt
vagnen själv backas upp på hjälpspår 1. Detta betyder att vagnen som ska
stå först i det färdiga tåget nu står först på hjälpspår 1. Resten av
vagnarna på huvudspåret backas nu upp på hjälpspår 2, alla vagnar från
hjälpspår 1 rullas fram på huvudspåret varvid alla vagnar från spår 2
åter rullas upp på huvudspåret. Detta har som följd att den sökta första
vagnen från det färdiga tåget nu står först på huvudspåret. Vagnsekvensen
efter är oviktig, då funktionen är rekursiv och nu går vidare och behandlar
nästa vagn som om det vore den första vagnen. Alla vagnar som är färdiga
kommer alltså stå kvar på huvudspåret, och algoritmen kommer att operera på
tågets svans. När inga vagnar återstår att behandla är funktionen klar.

\paragraph\
I min lösing använde jag mig av en hjälpfunktion, \emph{split}, som delar
upp ett tåg vid en given vagn, som beskrivet ovan. En serie om fyra
förflyttningar konstrueras sedan genom att, som beskrivet, flytta de olika
delarna av det uppdelade tåget för att nå det önskade läget; tågets första
vagn på rätt position. Denna serie förflyttningar byggs sedan på genom
att funktionen anropar sig själv med de efterföljande vagnarna som
``ursprungståg'' och återstående vagnar i det slutgiltiga tåget.

\begin{verbatim}
find([], []) ->
    [];
find(Train1, [H2 | T2]) ->
    {H1, T1} = split(Train1, H2),
    H1Len = length(H1),
    T1Len = length(T1),
    [
        {one, T1Len + 1},
        {two, H1Len},
        {one, -(T1Len + 1)},
        {two, -H1Len}
    ] ++ find(T1 ++ H1, T2).
\end{verbatim}

\subsection{Färre förflyttningar}
\paragraph\
Denna deluppgift, som gick ut på att optimera find-funktionen till att inte
flytta en vagn som redan befinner sig på rätt plats, kunde lösas väldigt
enkelt genom att lägga till följande funktionsklausul till funktionen från
den tidigare uppgiften:

\begin{verbatim}
few([H2 | T1], [H2 | T2]) ->
    few(T1, T2);
\end{verbatim}

Observera att funktionen här har bytt namn till \emph{few} istället för
\emph{find}. Denna nya funktionsklausul gör att om samma element ligger först
i både vagnslistan för ursprungståget och det slutgiltiga tåget behöver det
inte växlas fram och tillbaka, utan hoppas över och istället behandlas nästa
vagn i listan.

\subsection{Komprimeringar}
\paragraph\
Den sista uppgiften gick ut på att skriva en funktion som kompakterar en lista
med förflyttningar till en optimalare lista med färre funktioner.
Detta skulle lösas med hjälp av två regler:

\begin{itemize}
    \item Två direkt följande förflyttningar till samma spår kan slås samman
        till en större förflyttning som flyttar alla vagnar i ett drag.
    \item En förflyttning av noll vagnar medför ingen operation och kan därför
        tas bort.
\end{itemize}

\paragraph\
Implementationen går igenom listan med förflyttningar och utövar ovan regler
på varje elementpar om möjligt. Detta görs om igen rekursivt tills dess att
listan inte förändras längre.

\section{Utvärdering}

\paragraph\
Uppgiften medförde inte så mycket som kunde utvärderas mer än att given
indata matchar förväntad utdata. Detta kunde göras genom att låta
funktionen \emph{move} från deluppgift 3 utföra en sekvens flyttinstruktioner
från \emph{find} eller \emph{few} och se att det önskade tåget faktiskt
konstruerats på huvudspåret.

\paragraph\
För att testa detta skrev jag en liten testfunktion som skapar ett tåg av
valfri längd vilket ställs på huvudspåret.
Ett annat tåg, med samma vagnar fast i annan, slumpad ordning skapas och
används som det ``färdiga'' tåget som ska byggas genom att flytta vagnarna.
Funktionserna \emph{find} och \emph{few} anropas för att generera varsin
lista av instruktioner för att lösa uppgiften, och dessa listor skickas till
\emph{move} som utför dem. Resultatet (sista tillståndet i den returnerade
listan från \emph{move}) jämförs sedan med målet; det färdiga tåget; för att
se att korrekt resultat erhölls.

\paragraph\
I alla de tester jag gjorde, med tåg innehållandes 0 upp till 26 vagnar gav
både \emph{find} och \emph{few} väntat resultat.

\section{Sammanfattning}

\paragraph\
Sammanfattningsvis kan sägas att jag tyckte att uppgiften gick bra, men
att den var en aning tråkig; det var allderles för mycket förklarat
steg för steg i uppgifterna, vilket gjorde att det var mer mekaniskt
arbete än tankekraft som behövdes för att nå en lösning. Till exempel
tycker jag att föregående uppgift om Huffmankodning var betydligt intressantare
och klurigare, vilket naturligtvis gör uppgiften roligare.

\paragraph\
Jag tycker också att uppgiften var något svårläst, med många icke-
förklarande en- och tvåbokstävers variabelnamn som bidrog till en del
förvirring.

\end{document}
