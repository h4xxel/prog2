%% En enkel mall för att skapa en labb-raport.
\documentclass[a4paper, 11pt]{article}
\usepackage[utf8]{inputenc} 
\usepackage[swedish]{babel}
\usepackage{verbatim}
\usepackage{listings}
\usepackage{color}

\definecolor{dkblue}{rgb}{0,0,0.5}
\definecolor{dkyellow}{rgb}{0.5,0.5,0}
\definecolor{gray}{rgb}{0.5,0.5,0.5}
\definecolor{dkgreen}{rgb}{0,0.5,0}
\definecolor{dkmagenta}{rgb}{0.5,0,0.5}

\lstset{ %
    backgroundcolor=\color{white},   % choose the background color; you must add \usepackage{color} or \usepackage{xcolor}
    basicstyle=\footnotesize,        % the size of the fonts that are used for the code
    breakatwhitespace=false,         % sets if automatic breaks should only happen at whitespace
    breaklines=true,                 % sets automatic line breaking
    captionpos=b,                    % sets the caption-position to bottom
    commentstyle=\color{dkgreen},      % comment style
    deletekeywords={},            % if you want to delete keywords from the given language
    escapeinside={\%*}{*)},          % if you want to add LaTeX within your code
    extendedchars=true,              % lets you use non-ASCII characters; for 8-bits encodings only, does not work with UTF-8
    frame=none,                      % adds a frame around the code
    keepspaces=true,                 % keeps spaces in text, useful for keeping indentation of code (possibly needs columns=flexible)
    keywordstyle=\bfseries\color{dkblue},       % keyword style
    %identifierstyle=\color{dkyellow},
    language=erlang,                 % the language of the code
    morekeywords={*,...},            % if you want to add more keywords to the set
    numbers=left,                    % where to put the line-numbers; possible values are (none, left, right)
    numbersep=5pt,                   % how far the line-numbers are from the code
    numberstyle=\tiny\color{gray},   % the style that is used for the line-numbers
    rulecolor=\color{black},         % if not set, the frame-color may be changed on line-breaks within not-black text (e.g. comments (green here))
    showspaces=false,                % show spaces everywhere adding particular underscores; it overrides 'showstringspaces'
    showstringspaces=false,          % underline spaces within strings only
    showtabs=false,                  % show tabs within strings adding particular underscores
    stepnumber=1,                    % the step between two line-numbers. If it's 1, each line will be numbered
    stringstyle=\color{dkmagenta},       % string literal style
    tabsize=4,                       % sets default tabsize to 2 spaces
    title=\lstname                   % show the filename of files included with \lstinputlisting; also try caption instead of title
}

\renewcommand{\figurename}{Tabell }
\textwidth 5.1in

\title{Inlämningsuppgift Laboration 6}
\author{Axel Isaksson}
\date{2013-12-19} 

\begin{document}

\maketitle 

\section{Uppgiften}

\paragraph\
Veckans uppgift, och den sista för terminen gick ut på att implementera en så
kallad \emph{shoutcast}-server. Shoutcast är en utökning av
\emph{HTTP}-protokollet designat för att strömma musik och internetradio.

\paragraph\
Uppgiften bestod av flera delar, varav den sista delen lämnade frihet för egna
utökningar av den enkla shoutcastservern. De tidigare uppgifterna behandlade
tolkning av HTTP-protokollet, implementationsspecifik metadata för shoutcast
samt inläsning av \emph{ID3}-taggar från mp3-filer, för att kunna presentera
spårmetadata så som spårtitel, artist och album.

\section{Ansats}

\subsection{Shoutcast, \emph{ICY}}
\paragraph\
I första deluppgiften implementerades en HTTP-tolk och -kodare. Denna
konstruerades för att kunna hantera inramningsfel som sker när man skickar
data över \emph{tcp} -- ett datablock kan styckas upp och anlända i flera
mindre bitar som måste lappas ihop av mottagaren. Mycket kod för den denna
uppgift var redan given; tolken kan be om mer data från socketen, och genom
att dessutom skicka tillbaka en högre ordnings-funktion kan den återuppta
tolkningen av en förfrågan från den avbrutna positionen om ett ej helt block
läses in.

\paragraph\
De protokollspecifika detaljer för ICY-protokollet (\emph{I Can Yell},
internt namn för shoutcast) som skiljer sig från HTTP-standarden är att en
shoutcastkompatibel klient skickar med en specifik metadata-sträng i
meddelandehuvudet,
\begin{verbatim}
    Icy-MetaData: 1
\end{verbatim}
som talar om för servern att den
hanterar shoutcast-strömmar. Servern skickar i sin tur med strängen
\begin{verbatim}
    icy-metaint: BLOCKSIZE
\end{verbatim}
i sitt svar, där \emph{BLOCKSIZE} är datablockstorlek i ljudströmmen. 
Implementationen använder en blockstorlek på 8192 byte och servern skickar
i meddelandekroppen för svaret block efter block med mp3-data.
Mellan varje datablock skickas ett
litet block med metadata, se tabell \ref{tab:meta}. I detta metadata-block
skickas bland annat strömmens titel kontinuerligt som en textsträng:
\begin{verbatim}
StreamTitle='TITEL';
\end{verbatim}

\begin{table}
\centering
\begin{tabular}{|l|r|r|}  
\hline
Längd (byte) & 1 & variabel, 16$\times$längdfältet\\
\hline
Beskrivning & Längd för följande metadata & Datasträng, nollutfylld\\
\hline
\end{tabular}
\caption{Insprängd shoutcast-metadata}
\label{tab:meta}
\end{table}

\subsection{ID3-taggar}
\paragraph\
Nästa deluppgift gick ut på att extrahera metadata från ID3-taggen i en
mp3-fil. Implementationen hanterar bara ID3 version 1. För att vara
kompatibel med gamla mp3-spelare utan ID3-stöd ligger ID3-taggen som ett
128 byte stort block i slutet av filen. Blocket börjar med en trebokstävers
sträng, \emph{TAG}, som identifierare och är sedan indelat i block om 30
byte för titel, artist och album, följt av ett albumår på fyra byte, ett
kommentarsfält på ytterligare 30 samt ett genre-fält på en byte.
I ID3 version 1.1 är de två sista byten av kommentarsfältet reserverade
som en fast nollbyte samt ett spårnummerfält på en byte.

\paragraph\
Ett par funktioner skrevs för att öppna en mp3-fil, extrahera ID3-taggen
samt resten av datan. ID3-taggen delas upp som en lista av nyckel-värdepar
av i stil med \emph{\{title, ``Track title''\}}.

\subsection{En enkel server}
\paragraph\
Den första implementationen är väldigt grundläggande.
Här läses en mp3-fil in i början av körningen där titelfältet plockas ut
från ID3-taggen. Resten av mp3-datan styckas upp i 8192 byte stora block
där det sista, ofullständiga blocket (som ändå är knappt hörbart) förkastas
av enkelhetsskäl. En socket öppnas nu där servern lyssnar efter anslutningar
och accepterar den första möjliga. Servern läser och tolkar en HTTP-förfrågan
från klienten, och skickar till svar ett HTTP-meddelandehuvud tillbaka.
Efter meddelandehuvudet skickar servern block efter block från mp3-filen,
varvat med ett metadatablock innehållandes filens spårtitel. Slutligen stängs
anslutningen.

\subsection{Utökningar och förbättringar}
\paragraph\
För att utöka den grundläggande servern valde jag att implementera ett
webbgränssnitt för val av sång, som dessutom skulle kunna användas av flera
samtidigt. Då servern är byggd för att använda sockets i aktivt läge, där
systemet skickar meddelanden till processen som öppnade socketen vid inkommande
data, valde jag att använda mig av en pott med hanteringstrådar, där en 
inkommande anslutning tilldelas en tråd för mitt fleranvändarstöd.
Detta fungerar så att varje hanteringstråd kan vänta på accept från samma
lyssnings-socket, men bara en av trådarna kommer att tilldelas en ny,
inkommande anslutning.

\paragraph\
För att generera en lista över tillgängliga låtar använde jag mig av funktionen
\emph{file:list\textunderscore dir()} tillsammas med \emph{lists:foldl()}
för att lista
innehållet i musikmappen, och filtrera ut alla filer med ändelsen \emph{.mp3}.
För varje mp3-fil plockas metadatan i ID3-taggen ut och en html-tabell skapas,
sorterad efter Artist, Album, Spårnummer med alla låtar. Sorteringen sker dock
på hela HTML-strängen, vilket inte är optimalt, men enkelt att implementera och
betydligt bättre än en osorterad lista.

\begin{lstlisting}[language=erlang]
list_directory(Path) ->
  {ok, Dir} = file:list_dir(Path),
  Mp3s = lists:foldl(fun(E, Acc) ->
    case lists:reverse(E) of
      [$3, $p, $m, $. | _] ->
        Filename = E,
        Id3 = mp3:read_tags(Path ++ "/" ++ urldecode(Filename)),
        ["<tr>\n" ++
          "<td>" ++ mp3:id3_field(artist, Id3) ++ "</td>\n" ++
          "<td>" ++ mp3:id3_field(album, Id3) ++ "</td>\n" ++
          "<td>" ++ mp3:id3_field(track, Id3) ++ "</td>\n" ++
          "<td>" ++ generate_url(mp3:id3_field(title, Id3), Filename) ++ "</td>\n" ++
        "</tr>\n" | Acc];
      _ ->
        Acc
    end
  end, [], Dir),
  List = lists:flatten(lists:sort(Mp3s)),
  Response = list_to_binary("<!DOCTYPE html>\n<html>\n<head>\n<title>Streaming server</title>\n</head>\n<body>\n" ++
    "<h1>Streaming server</h1>\n" ++
    "<table border=\"1\">\n<tr><th>Artist</th><th>Album</th><th>Track</th><th>Title</th></tr>\n" ++
    List ++ "</table>\n</body>\n</html>\n"),
  {dir, byte_size(Response), Response}.
\end{lstlisting}

\paragraph\
Servern utökades till att kunna tolka den efterfrågade resursen från
GET-metoden i klientens HTTP-förfrågan. För resursnamn som börjar på
\emph{/mp3/} letar servern efter resten av resursnamnet i serverns
musikmapp. Förfrågan efter webbroten \emph{/} returnerar html-sidan
med låtlistan, övriga resurser returnerar HTTP-felmeddelande 404.
När en fil från musikkatalogen efterfrågas läses den in och styckas
upp för att skickas iväg till klienten. Jag implementerade även stöd
för att kunna strömma till klienter som inte hanterar shoutcast, genom
att inte skicka inbakad metadata om inte klienten specifikt ber om den,
då den annars tolkas som mp3-data och leder till hackig uppspelning.

\section{Utvärdering}

\paragraph\
Denna enkla server fungerar för mindre tillämpningar med lämnar en del
utrymme för förbättring. En möjlig utökning kunde vara möjligheten
att spela upp ett helt album, eller en spellista åt gången. Detta skulle
kunna realiseras genom att låta servern "klistra ihop" flera mp3-filer i
minnet genom att öppna nästa fil och börja skicka block från den direkt
efter att den första filen skickats. Här kan man också passa på att läsa
in den nya sångtiteln från ID3-taggen och skicka med den med de nya blocken.

\paragraph\
En annan möjlig förbättring är att bygga en HTML-motor som genererar HTML
utifrån en trädstruktur som beskriver sidan, istället för att klippa och
klistra ihop textsträngar. Detta gör det mycket enklare att utöka serverns
gränssnitt, man kan även använda mallar för att göra det enkelt att ändra
utseendet, samtidigt som man ser till att följa gällande HTML-standard.
Detta gör det även relativt enkelt att generera en inloggningssida.
Genom att kontrollera inloggningsuppgifter mot användarnamn och
lösenordshash på servern, tillsamans med sessionskakor kan en mer
``personlig'' sida skapas där användaren kan spara sida spellistor
i en databas på servern.

\section{Sammanfattning}

\paragraph\
Sammanfattningsvis kan sägas att denna uppgift var såväl rolig och
intressant som lärorik. De störe problemen som stöttes på längs vägen
var bland annat att sockets i aktivt läge bara skickar meddelanden till
tråden som öppnade socketen genom accept, och inte tråden som sätter
den i aktivt läge genom \emph{inet:setsocketopt()}, vilket ledde till att
jag valde att implementera en poolad server istället för att skapa en ny
tråd för varje anslutning. Andra, smärre problem som upptod var vid
inläsning av ID3-taggarna då olika datatyper hanterar sin storlek olika
(i bitar eller byte), samt att storleken för den insprängda medadatan
räknades ut felaktigt, vilket ledde till hackig uppspelning. Dessa problem
löstes emellertid enkelt genom en titt i dokumentationen för uppfriskning av 
minnet.

\end{document}
